\documentclass[pdf]{beamer}
\mode<presentation>{}
\usepackage{wrapfig}
\usepackage{setspace}
\usepackage{graphicx}
\usepackage{amsmath,amssymb, amsthm}
\usepackage{subfig}
\usepackage{framed}
\usepackage{enumerate}
\usepackage{pdfsync}
\usepackage{bbm}
\usepackage{float}
\usepackage{dsfont}
\usetheme{Copenhagen}

\title{R Examples from BIOST 511 - Set 7 and 8}
\subtitle{Taught by Lurdes Inoue, Ph.D.}
\author{Brian D. Williamson}
\institute{University of Washington \\ Department of Biostatistics}
\date{\today}
\begin{document}

\begin{frame}
\titlepage
\end{frame}

\begin{frame}
\frametitle{Disclaimer}
All of the following slides and functions are to be used when given sample descriptive statistics

For raw data, refer to the documentation for \texttt{ttest()} and other functions given on http://www.emersonstatistics.com/R
\end{frame}
\section{One Sample Tests}
\begin{frame}[fragile]
\frametitle{Test of Proportions}
Use the \texttt{prop.test()} function from the \texttt{stats} package (automatically loaded into R along with the base package)
{\fontsize{6pt}{7.2}\selectfont
\begin{verbatim}
prop.test(110, 200, p=.3, alternative="two.sided")

1-sample proportions test with continuity correction

data:  110 out of 200, null probability 0.3
X-squared = 58.3393, df = 1, p-value = 2.206e-14
alternative hypothesis: true p is not equal to 0.3
95 percent confidence interval:
 0.4782703 0.6197775
sample estimates:
   p 
0.55 
\end{verbatim}}
-- If we want a different test, we change the \texttt{alternative} argument
\end{frame}

\begin{frame}[fragile]
\frametitle{t-test using Sample Descriptive Statistics}
Use the \texttt{ttesti()} function from the \texttt{uwIntroStats} package
{\fontsize{6pt}{7.2}\selectfont
\begin{verbatim}
ttesti(24, 175, 35, 43, null.hyp=230)

 One-sample t-test 
  Obs Mean Std. Error SD 95 %CI         
x 24  175  7.1443     35 160.221 189.779

 t-statistic = -7.698396 , df = 23 

 Ho: mean =  230 

 Ha: mean != 230 , Pr(|T| > |t|) = 8.24e-08 
\end{verbatim}}
-- Again, if we want a different test, we change the \texttt{alternative} argument
\end{frame}

\begin{frame}[fragile]
\frametitle{Binomial Test}
Use the \texttt{binom.test()} function from the \texttt{stats} package
{\fontsize{6pt}{7.2}\selectfont
\begin{verbatim}
binom.test(110, 200, p=.3, alternative="two.sided")

	Exact binomial test

data:  110 and 200
number of successes = 110, number of trials = 200, p-value = 3.125e-13
alternative hypothesis: true probability of success is not equal to 0.3
95 percent confidence interval:
 0.4782493 0.6202464
sample estimates:
probability of success 
                  0.55 
\end{verbatim}}
-- Again, if we want a different test, we change the \texttt{alternative} argument
\end{frame}

\begin{frame}[fragile]
\frametitle{Group 1 and Group 2 One-Sample Tests}
{\fontsize{6pt}{7.2}\selectfont
\begin{verbatim}
## One sample t-test for Group 1
ttesti(obs=10, mean=-1.6, sd=1.5, null.hyp=0)

 One-sample t-test 
  Obs Mean Std. Error SD  95 %CI       
x 10  -1.6 0.4743     1.5 -2.673 -0.527

 t-statistic = -3.373096 , df = 9 

 Ho: mean =  0 

 Ha: mean != 0 , Pr(|T| > |t|) = 0.0082165 
## For Group 2
ttesti(obs=30, mean=-.7, sd=2.1, null.hyp=0)

 One-sample t-test 
  Obs Mean Std. Error SD  95 %CI      
x 30  -0.7 0.3834     2.1 -1.484 0.084

 t-statistic = -1.825742 , df = 29 

 Ho: mean =  0 

 Ha: mean != 0 , Pr(|T| > |t|) = 0.078203 
\end{verbatim}}
\end{frame}

\section{Two Sample Tests}
\begin{frame}[fragile]
\frametitle{Difference of Means with Sample Descriptive Statistics}
\framesubtitle{Unequal Variances}
Use the \texttt{ttesti()} function in the \texttt{uwIntroStats} package

For the Group 1 Group 2 example,
{\fontsize{6pt}{7.2}\selectfont
\begin{verbatim}
## Two sample for the difference of means, with variance unequal
ttesti(10, -1.6, 1.5, 30, -.7, 2.1)

 Two-sample t-test with unequal variances 
     Obs Mean Std. Error SD  95 %CI       
x    10  -1.6 0.4743     1.5 -2.673 -0.527
y    30  -0.7 0.3834     2.1 -1.484 0.084 
diff 40  -0.9 0.6099     NA  -2.135 0.335 

 t-statistic = -1.475608 , Satterthwaite's df = 21.72386 

 Ho: mean(x) - mean(y) = diff =  0 

 Ha: diff != 0 , Pr(|T| > |t|) = 0.1544 
\end{verbatim}}
\end{frame}

\begin{frame}[fragile]
\frametitle{Proportions}
Use the \texttt{binom.test()} function in the \texttt{stats} package

{\fontsize{6pt}{7.2}\selectfont
\begin{verbatim}
## Exact binomial test
binom.test(7, 26, .5)

	Exact binomial test

data:  7 and 26
number of successes = 7, number of trials = 26, p-value = 0.02896
alternative hypothesis: true probability of success is not equal to 0.5
95 percent confidence interval:
 0.1157322 0.4778748
sample estimates:
probability of success 
             0.2692308 
\end{verbatim}}
\end{frame}

\section{Tests for Categorical Data}
\begin{frame}[fragile]
\frametitle{Chi-squared Test}
Use the \texttt{chisq.test()} function in the \texttt{stats} package

If given raw data, use the \texttt{tabulate()} function in the \texttt{uwIntroStats} package

The \texttt{correct=FALSE} argument tells R to NOT use Yates' continuity correction

For this small data example, you will get a warning from R, because the expected cell counts are too small for a reliable Chi-squared test

{\fontsize{6pt}{7.2}\selectfont
\begin{verbatim}
chisq.test(matrix(c(2,23,5,30), nrow=2, byrow=T), correct=FALSE)

	Pearson's Chi-squared test

data:  matrix(c(2, 23, 5, 30), nrow = 2, byrow = T)
X-squared = 0.5591, df = 1, p-value = 0.4546
\end{verbatim}}
\end{frame}

\begin{frame}[fragile]
\frametitle{Fisher's Exact Test}
Use the \texttt{fisher.test()} function in the \texttt{stats} package

If given raw data, use the \texttt{tabulate()} function in the \texttt{uwIntroStats} package with the correct arguments

{\fontsize{6pt}{7.2}\selectfont
\begin{verbatim}
fisher.test(matrix(c(2,23,5,30), nrow=2, byrow=T))

	Fisher's Exact Test for Count Data

data:  matrix(c(2, 23, 5, 30), nrow = 2, byrow = T)
p-value = 0.6882
alternative hypothesis: true odds ratio is not equal to 1
95 percent confidence interval:
 0.04625243 3.58478157
sample estimates:
odds ratio 
  0.527113 

\end{verbatim}}
\end{frame}

\begin{frame}[fragile]
\frametitle{Chi-squared Goodness of Fit Test}
Use the \texttt{chisq.test()} function in the \texttt{stats} package

Simply enter in a vector rather than a matrix and a goodness of fit test will be performed

\end{frame}

\begin{frame}[fragile]
\frametitle{Homogeneity Test}
Use the \texttt{chisq.test()} function in the \texttt{stats} package

If given raw data, use the \texttt{tabulate()} function in the \texttt{uwIntroStats} package with the correct arguments

{\fontsize{6pt}{7.2}\selectfont
\begin{verbatim}
## Chi-squared example
chisq.test(matrix(c(7, 55, 489, 475, 
                    293, 38, 61, 129, 570, 431, 154, 12), 
                  nrow=2, byrow=T))

	Pearson's Chi-squared test

data:  matrix(c(7, 55, 489, 475, 293, 38, 61, 129, 570, 431, 154, 12), nrow = 2, byrow = T)
X-squared = 137.7193, df = 5, p-value < 2.2e-16                  
\end{verbatim}}
\end{frame}

\begin{frame}[fragile]
\frametitle{Independence Test}
{\fontsize{6pt}{7.2}\selectfont

\begin{verbatim}
chisq.test(matrix(c(52, 79, 342, 226, 62, 153, 417, 262, 53, 213, 629, 375, 
                    54, 231, 571, 244, 18, 46, 139, 74, 25, 139, 330, 116), 
                  nrow=4, byrow=T))
	Pearson's Chi-squared test

data:  matrix(c(52, 79, 342, 226, 62, 153, 417, 262, 53, 
213, 629, 375, 54, 231, 571, 244, 18, 46, 139, 74, 25, 139, 330, 116), nrow = 4, byrow = T)
X-squared = 2041.811, df = 15, p-value < 2.2e-16
                  
\end{verbatim}}
\end{frame}

\section{Nonparametric Tests}
\begin{frame}[fragile]
\frametitle{Sign Test}
No direct analog for this test in R

Since it ignores a lot of information, use the binomial test or McNemar's test
\end{frame}
\begin{frame}[fragile]
\frametitle{Wilcoxon Signed Rank Test}
Use the \texttt{wilcoxon} function from the \texttt{uwIntroStats} package

First you must have data vectors, so create them if necessary

{\fontsize{6pt}{7.2}\selectfont
\begin{verbatim}
cf <- c(1153, 1132, 1165, 1460, 1162, 1493, 1358, 1453, 1185, 1824, 1793, 1930, 2075)
healthy <- c(996, 1080, 1182, 1452, 1634, 1619, 1140, 1123, 1113, 1463, 1632, 1614, 1836)
wilcoxon(cf, healthy, paired=TRUE)
 
 Wilcoxon signed rank test 
         obs sum ranks expected
positive  10        71     45.5
negative   3        20     45.5
zero       0         0      0.0
all       13        91     91.0
                            
unadjusted variance   204.75
adjustment for ties     0.00
adjustment for zeroes   0.00
adjusted variance     204.75
                    H0 Ha       
Hypothesized Median 0  two.sided
  Test Statistic p-value 
V 71             0.080322
Z 1.7821         0.037368
\end{verbatim}}
\end{frame}

\begin{frame}[fragile]
\frametitle{Wilcoxon Rank Sum Test}
Use the \texttt{wilcoxon} function from the \texttt{uwIntroStats} package

First you must have data vectors, so create them if necessary

{\fontsize{6pt}{4.2}\selectfont
\begin{verbatim}
ebv <- c(2.9, 12.1, 2.6, 2.5, 2.8, 15.8, 3.2, 1.8, 7.8, 
         2.9, 3.2, 8.0, 1.5, 6.3, 1.2, 3.5, 4.5, 1.3, 
         1.0, 1.0, 1.3, 1.9, 1.3, 2.1, 2.1, 1.0) 
sero <- c(rep(1,16), rep(0,10))
data <- cbind(ebv, sero)
wilcoxon(data[sero==1], data[sero==0])

 Wilcoxon rank sum test 
         obs rank sum expected
Y         20      369      689
X         32     1009      689
combined  52     1378     1378
                               
unadjusted variance 2826.666667
adjustment for ties    2.968326
adjusted variance   2823.698341
                    H0 Ha       
Hypothesized Median 0  two.sided
  Test Statistic p-value   
W 481            0.0018285 
Z 3.1168         0.00091424
Warning message:
In wilcoxon.do(x = y, y = x, alternative = alternative, mu = mu,  :
  cannot compute exact p-value with ties
\end{verbatim}}
The warning message means that there are tied values, so the p-values are approximate
\end{frame}

\section{Simple Linear Regression}
\begin{frame}[fragile]
\frametitle{Systolic Blood Pressure Example}
Create the data set by saving it as a text file, then load it into R and attach it
{\fontsize{6pt}{4.2}\selectfont
\begin{verbatim}
systolic <- read.table("P:/TA/systolic.txt", header=TRUE, quote="\"", stringsAsFactors=FALSE)
\end{verbatim}}

Next regress \texttt{systolic} (the y-variable) vs \texttt{age} using the \texttt{regress()} function, with first argument \texttt{"mean"}
{\fontsize{6pt}{4.2}\selectfont
\begin{verbatim}
regress("mean", systolic$systolic, model=uModel(systolic$age))

Call:
regress(fnctl = "mean", y = systolic$systolic, model = uModel(systolic$age))

Residuals:
    Min      1Q  Median      3Q     Max 
-7.1395 -2.3314 -0.2163  2.1872  5.8605 

Coefficients:
                Estimate  Naive SE  Robust SE    95%L      95%H         F stat    df Pr(>F)   
Intercept         67.68     3.191     2.440        62.45     72.91         769.25 1  < 0.00005
systolic$age      6.153     0.9283    0.6511       4.757     7.550          89.32 1  < 0.00005

Residual standard error: 3.403 on 14 degrees of freedom
Multiple R-squared:  0.7584,	Adjusted R-squared:  0.7411 
F-statistic: 43.94 on 1 and 14 DF,  p-value: 1.135e-05
\end{verbatim}}

The \texttt{uModel()} function gives the correct names and allows multiple variables to be listed in the \texttt{model} argument
\end{frame}

\begin{frame}[fragile]
\frametitle{Prediction of the mean}
Again use the \texttt{regress()} function

We will need items from the list it returns, so get the predicted $\hat{y}$'s with \texttt{\$linearPredictor} and the standard deviation with \texttt{\$sigma}:
{\fontsize{5pt}{5.2}\selectfont
\begin{verbatim}
yhat <- sysRegress$linearPredictor
## get standard error about line
se <- sysRegress$sigma/sqrt(14)
## calculate confidence interval around the line using the quantile of the t distribution
lower <- yhat - qt(.975, 14)*se
upper <- yhat + qt(.975, 14)*se
cbind(round(lower, 3), round(yhat, 3), round(upper,3))
        [,1]   [,2]    [,3]
 [1,] 84.189 86.140  88.090
 [2,] 90.342 92.293  94.244
 [3,] 84.189 86.140  88.090
 [4,] 78.035 79.986  81.937
 [5,] 90.342 92.293  94.244
 [6,] 96.496 98.447 100.397
 [7,] 78.035 79.986  81.937
 [8,] 84.189 86.140  88.090
 [9,] 96.496 98.447 100.397
[10,] 90.342 92.293  94.244
[11,] 78.035 79.986  81.937
[12,] 84.189 86.140  88.090
[13,] 84.189 86.140  88.090
[14,] 90.342 92.293  94.244
[15,] 84.189 86.140  88.090
[16,] 84.189 86.140  88.090
\end{verbatim} }
\end{frame}

\begin{frame}[fragile]
\frametitle{Prediction of a new value}
Again use the \texttt{regress()} function
{\fontsize{5pt}{5.2}\selectfont
\begin{verbatim}
yhat <- sysRegress$linearPredictor
## get standard error about line
se <- sysRegress$sigma*sqrt(1+1/16+(systolic$age-mean(systolic$age))^2/(sd(systolic$age)^2))
## calculate confidence interval around the line using the quantile of the t distribution
lower <- yhat - qt(.975, 14)*se
upper <- yhat + qt(.975, 14)*se
cbind(round(lower, 3), round(yhat, 3), round(upper,3))
        [,1]   [,2]    [,3]
 [1,] 78.240 86.140  94.039
 [2,] 83.090 92.293 101.497
 [3,] 78.240 86.140  94.039
 [4,] 67.375 79.986  92.597
 [5,] 83.090 92.293 101.497
 [6,] 83.416 98.447 113.478
 [7,] 67.375 79.986  92.597
 [8,] 78.240 86.140  94.039
 [9,] 83.416 98.447 113.478
[10,] 83.090 92.293 101.497
[11,] 67.375 79.986  92.597
[12,] 78.240 86.140  94.039
[13,] 78.240 86.140  94.039
[14,] 83.090 92.293 101.497
[15,] 78.240 86.140  94.039
[16,] 78.240 86.140  94.039
\end{verbatim} }
\end{frame}
\end{document}