\documentclass[pdf]{beamer}
\mode<presentation>{}
\usepackage{wrapfig}
\usepackage{setspace}
\usepackage{graphicx}
\usepackage{amsmath,amssymb, amsthm}
\usepackage{subfig}
\usepackage{framed}
\usepackage{enumerate}
\usepackage{pdfsync}
\usepackage{bbm}
\usepackage{float}
\usepackage{dsfont}
\usetheme{Copenhagen}

\title{R Examples from BIOST 514/517}
\subtitle{Taught by Katie Kerr, Ph.D.}
\author{Brian D. Williamson}
\institute{University of Washington \\ Department of Biostatistics}
\date{\today}
\begin{document}

\begin{frame}
\titlepage
\end{frame}

\section{Lecture 13-16}
\begin{frame}[fragile]
\frametitle{Difference in Means}
Default in R assumes variances unequal
Can use \texttt{t.test()} (\texttt{stats}) or \texttt{ttest()} (\texttt{uwIntroStats}). You can set \texttt{var.eq=T} for equal variances

Make sure to load all of the required packages and data
\framesubtitle{FEV data}
{\fontsize{6pt}{5}\selectfont
\begin{verbatim}
smoke <- SMOKE
sex <- SEX

## ttest for difference in means
height <- HEIGHT
ttest(height, by=sex)

Call:
ttest.default(var1 = height, by = sex)

Two-sample t-test allowing for unequal variances :
 
Summary:
        Group Obs Missing  Mean Std. Err. Std. Dev.          95% CI
      sex = 2 336       0 62.03     0.345      6.33  [61.346, 62.7]
      sex = 1 318       0 60.21     0.269      4.79 [59.683, 60.74]
   Difference 654       0  1.81     0.438      <NA>   [0.954, 2.67]

 Ho: difference in  means = 0 ; 
 Ha: difference in  means != 0 
 t = 4.144 , df = 622 
 Pr(|T| > t) =  3.88659e-05 
\end{verbatim}}
\end{frame}

\begin{frame}[fragile]
\frametitle{Proportions}
\framesubtitle{FEV data}
Calculate the confidence interval by yourself! Use the \texttt{qt()} to get $t$ quantiles, and \texttt{qbinom()} for binomial quantiles
\end{frame}

\begin{frame}[fragile]
\frametitle{Comparing Proportions}
\framesubtitle{FEV data}
\texttt{prop.test()} function

Gives us a two-sided test by default (change with \texttt{alternative} option). Tests equality of the proportions
{\fontsize{6pt}{7.2}\selectfont
\begin{verbatim}
## difference of proportions
is_smoker <- smoke==1
successMale <- smoke==1 & sex==1
table(successMale)
prop1 <- 26
## recall that we already know the observations from the t-test above
obs1 <- 336
successFemale <- smoke==1 & sex==2
prop2 <- 39
obs2 <- 318
prop.test(c(prop1, prop2), c(obs1, obs2), correct=FALSE)

	2-sample test for equality of proportions without continuity correction

data:  c(prop1, prop2) out of c(obs1, obs2)
X-squared = 3.739, df = 1, p-value = 0.05316
alternative hypothesis: two.sided
95 percent confidence interval:
 -0.0912611312  0.0007400171
sample estimates:
    prop 1     prop 2 
0.07738095 0.12264151 
\end{verbatim}}
\end{frame}

\begin{frame}[fragile]
\frametitle{Odds Ratios and Risk Ratios}
Download the \texttt{epitools} package, or calculate by hand

The \texttt{riskratio} and \texttt{oddsratio} functions will calculate for entered counts
\end{frame}

\begin{frame}[fragile]
\frametitle{One-sample t-test}
Given count data, use the \texttt{ttesti()} function from \texttt{uwIntroStats}

\begin{verbatim}
ttesti(25, 220, 46, null.hyp=211)

 One-sample t-test 
  Obs Mean Std. Error SD 95 %CI          
x 25  220  9.2        46 [201.01, 238.99]

 t-statistic = 0.978 , df = 24 

 Ho: mean =  211 

 Ha: mean != 211 , Pr(|T| > |t|) = 1.6623 
\end{verbatim}
\end{frame}

\begin{frame}[fragile]
\frametitle{Two-sample t-test}
\frametitle{WCGS Data}
{\fontsize{7pt}{7.2}\selectfont
\begin{verbatim}
ttest(sbp, by=dibpat)

Call:
ttest.default(var1 = sbp, by = dibpat)

Two-sample t-test allowing for unequal variances :
 
Summary:
            Group  Obs Missing   Mean Std. Err. Std. Dev.           95% CI
   dibpat = A1,A2 1589       0 129.78     0.394      15.7 [129.01, 130.55]
   dibpat = B3,B4 1565       0 127.47     0.365      14.4 [126.75, 128.18]
       Difference 3154       0   2.32     0.537      <NA>     [1.26, 3.37]

 Ho: difference in  means = 0 ; 
 Ha: difference in  means != 0 
 t = 4.317 , df = 3137 
 Pr(|T| > t) =  1.62861e-05 
\end{verbatim}}
\end{frame}

\begin{frame}[fragile]
\frametitle{Paired t-test}
\frametitle{Shoulder Pain Data}
Use the \texttt{matched=TRUE} option!

Can also run a one-sample t-test as above
{\fontsize{7pt}{7.2}\selectfont
\begin{verbatim}
ttest(pain[time==1], pain[time==6], matched=TRUE)

Call:
ttest.default(var1 = pain[time == 1], var2 = pain[time == 6], 
    matched = TRUE)

Two-sample (matched) t-test  :
 
Summary:
             Group Obs Missing  Mean Std. Err. Std. Dev.         95% CI
   pain[time == 1]  41       0 1.098     0.206     1.319 [0.681, 1.514]
   pain[time == 6]  41       0 0.585     0.131     0.836 [0.322, 0.849]
        Difference  41       0 0.512     0.185      1.17 [0.138, 0.886]

 Ho: difference in  means = 0 ; 
 Ha: difference in  means != 0 
 t = 2.766 , df = 40 
 Pr(|T| > t) =  0.00855172 
\end{verbatim}}
\end{frame}

\begin{frame}[fragile]
\frametitle{Goodness of Fit Test}
\frametitle{PSA Data}
Use the \texttt{tabulate} function

Use the same function in the PBC data set examples

{\fontsize{7pt}{7.2}\selectfont
\begin{verbatim}
tabulate(grade, bss)

Call:
tabulate.default(grade, bss)
            bss.1 bss.2 bss.3 bss.NA bss.NotNA bss.ALL
grade.1      1     3     6     0     10        10     
grade.2      2     4     9     0     15        15     
grade.3      2     4     9     1     15        16     
grade.NA     0     2     6     1      8         9     
grade.NotNA  5    11    24     1     40        41     
grade.ALL    5    13    30     2     48        50     
            Point Estimate Test Statistic df 95% CI p-value Warnings
Chi-squared                4.3115         16        0.99825    
\end{verbatim}}
\end{frame}

\begin{frame}[fragile]
\frametitle{Fisher's Exact Test}
Use the \texttt{fisher.test()} function, or the \texttt{tabulate()} function with the \texttt{tests="fisher"} option
\end{frame}

\begin{frame}[fragile]
\frametitle{Survival}
Create a \texttt{Surv} variable, plot the curve with confidence interval

To plot cumulative hazard function, use the \texttt{fun="cumhaz"} option

To compare survival times, use the \texttt{survdiff} function from the \texttt{survival} package
\begin{verbatim}
relapse <- ifelse(inrem=="no", 1, 0)
atrisk <- Surv(obstime, relapse)
survFit <- survfit(atrisk~1)
plot(survFit)
\end{verbatim}
\end{frame}
\end{document}