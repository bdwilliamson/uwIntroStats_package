\documentclass[pdf]{beamer}
\mode<presentation>{}
\usepackage{wrapfig}
\usepackage{setspace}
\usepackage{graphicx}
\usepackage{amsmath,amssymb, amsthm}
\usepackage{subfig}
\usepackage{framed}
\usepackage{enumerate}
\usepackage{pdfsync}
\usepackage{bbm}
\usepackage{float}
\usepackage{dsfont}
\usetheme{Copenhagen}

\title{R Examples from BIOST 514/517}
\subtitle{Taught by Katie Kerr, Ph.D.}
\author{Brian D. Williamson}
\institute{University of Washington \\ Department of Biostatistics}
\date{\today}
\begin{document}

\begin{frame}
\titlepage
\end{frame}

\section{Lecture 18}
\begin{frame}[fragile]
\frametitle{Sign Test}\
\framesubtitle{Shoulder Pain Data}
Use the \texttt{wilcoxon()} function from \texttt{uwIntrostats}, specifying that \texttt{paired=T}.
\begin{verbatim}
wilcoxon(pain1, pain6, paired=TRUE)
\end{verbatim} 
\end{frame}

\begin{frame}[fragile]
\frametitle{Mann-Whitney U Test/Rank-sum test}
\framesubtitle{PBC Data}
Use the \texttt{wilcoxon()} function from \texttt{uwIntroStats}. In the bilirubin example, need to break up by edema first
\begin{verbatim}
edema0 <- pbc[pbc$edema==0,"bilirubin"]
edema1 <- pbc[pbc$edema==1,"bilirubin"]
wilcoxon(edema0)
wilcoxon(edema1)
\end{verbatim}
\end{frame}

\end{document}